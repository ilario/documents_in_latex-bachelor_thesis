%%%%%%%% Capitolo4 %%%%%%%%
\chapter{Conclusioni}

Gli obiettivi di questo tirocinio di tesi sono stati raggiunti:
\begin{itemize}
 \item sono state sintetizzate e caratterizzate nanoparticelle di CdSe di dimensioni nanoscopiche,
 \item è stato sintetizzato e caratterizzato poli(3-esiltiofene) regioregolare e terminato con singolo gruppo di acido fosfonico,
 \item sono stati eseguiti e caratterizzati due scambi di leganti sulla superficie delle nanoparticelle.
\end{itemize}
Le nanoparticelle sintetizzate sono risultate avere diametro di 3.0~nm e sono state ottenute con la procedura messa a punto da Peng \etal  \cite{qd-CdSe-CdO}, decisamente più sicura per l'operatore rispetto alle procedure alternative. 

Il poli(3-esiltiofene) sintetizzato è risultato regioregolare \emph{head-tail-head-tail} e terminato in modo asimmetrico. Il metodo utilizzato è stato messo a punto da Jeffries-El \etal \cite{pol-p3ht-end} e risulta in una polimerizzazione quasi vivente, perciò con la possibilità di avere polimeri con bassa polidispersità. La lunghezza media ottenuta è stata di 23 monomeri. Lunghezze maggiori non avrebbero effetto sull'ampiezza del \emph{band gap} ma sarebbero preferibili per future applicazioni in celle fotovoltaiche (una maggiore lunghezza diminuirebbe il trasporto intercatena necessario per la collezione delle cariche sugli elettrodi). 

La molecola contenente un acido fosfonico protetto è stata legata con successo ad una sola terminazione del polimero e deprotetta rendendolo così un polimero legante di superficie nei confronti del CdSe.

Sono stati effettuati 2 scambi di leganti sulla superficie delle nanoparticelle: da una trialchil fosfina ossido ed un acido fosfonico a piridina e poi dalla piridina al polimero terminato con un acido fosfonico. Durante questi processi si è riusciti ad evitare l'aggregazione delle nanoparticelle. Nel primo scambio si è ottenuto, non senza difficoltà, di sostituire completamente i leganti precedentemente presenti. Nel secondo scambio di leganti {%\color{red}
 il polimero è stato quantitativamente legato alle nanoparticelle tramite la terminazione legante.}

Un ideale completamento di questo lavoro richiederebbe la realizzazione di una cella reale e la sua caratterizzazione elettrica {
che potrà essere effettuata in futuro grazie alla collaborazione con altri gruppi di ricerca.} 

{L'aspetto di maggiore innovatività di questo lavoro è stato quindi la sintesi e valutazione di un polimero coniugato contenente una terminazione in grado di coordinare tale polimero con la superficie 
di nanoparticelle inorganiche. Per ottenere efficienze energetiche maggiori nel campo del fotovoltaico ibrido molti sono i fattori su cui porre l'attenzione, ma il primo ed attualissimo problema è la realizzazione di una morfologia \emph{bulk heterojunction}. Si intuisce che l'utilizzo di un polimero legante potrebbe favorire notevolmente il mantenimento di una simile morfologia in una condizione di equilibrio. Va peraltro considerato che la scelta tra i possibili gruppi leganti non è di importanza secondaria perché questa terminazione legante dovrà avere il ruolo di contatto elettrico senza agire da barriera o trappola per i portatori di carica. Questo comportamento è difficile da prevedere a priori. In conclusione si rende necessario un grande lavoro di ricerca, ma questo era prevedibile vista la grande complessità del sistema che vogliamo ottimizzare.
}
