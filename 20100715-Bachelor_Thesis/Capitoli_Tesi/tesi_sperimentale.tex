\chapter{Parte sperimentale}

\section{Informazioni generali}
Per le analisi TLC sono state usate lastre di gel di silice Fluka con indicatore di fluorescenza.
Le purificazioni su colonna sono state eseguite con una fase stazionaria di Gel di silice 60 Fluka ed eluente una miscela composta da {\itshape n}-esano e tetraidrofurano in rapporto 60:40.
Gli spettri $^1$H-NMR, $^{13}$C-NMR, $^{31}$P $\{^1H\}$ NMR e 2D HMQC sono stati registrati con spettrometro Varian XL-300 da 300MHz con rotazione fissata su 20Hz.
Gli spettri IR sono stati registrati con uno spettrofotometro Perkin-Elmer Spectrum One depositando il campione in soluzione su una pasticca di KBr e lasciandolo seccare.
Gli spettri UV-vis sono stati realizzati con uno spettrofotometro Perkin-Elmer Lambda 650.
Gli spettri di fotoluminescenza sono stati registrati con uno spettrofotometro Perkin-Elmer LS 55.
Le analisi DLS (\emph{Dynamic Light Scattering}) sono state realizzate con un apparecchio Brookhaven 90plus.
I solventi toluene, piridina (anidrificata con idruro di calcio), cloroformio (anidrificato con solfato di sodio) e tetraidrofurano sono stati lasciati a riflusso per 7 ore al buio e distillati sotto azoto con apparato di Claisen prima dell’uso. Sono stati conservati sotto azoto con setacci molecolari 3A. 
Le filtrazioni sono state effettuate con filtro per siringa Millipore Millex-FG in PTFE (teflon) da 0.2$\mu$m.
Le centrifugazioni sono state eseguite utilizzando una centrifuga ALC 4236.
Ogni apparecchiatura anidra in atmosfera inerte è stata ottenuta con almeno 3 passaggi di gas inerte (azoto o argon), vuoto, riscaldamento con pistola termica.
Le soluzioni sensibili all’aria o all’umidità sono state trasferite utilizzando siringhe ipodermiche tipo Luer-Lock.

\section{Sintesi di \emph{quantum dots} di CdSe passivati con TOPO e TDPA}
In un pallone a 3 colli con agitazione è stata fatta atmosfera inerte di argon. Vi sono stati inseriti ossido di cadmio$_{(s)}$ ( CdO, 0.05123~g, 0.39896~mmol, Aldrich, 99.99\% ), acido 1-tetradecilfosfonico$_{(s)}$ ( TDPA, 0.223~g, 278.37~g/mol, 0.801~mmol, Alfa Aesar, 98\% ) e tri-ottilfosfina ossido$_{(s)}$ ( TOPO, 3.77855~g, 386.63~g/mol, 9.77291~mmol, Aldrich, 99\% ). Si è scaldato con un mantello riscaldante e nel frattempo il pallone è stato posto sotto vuoto da pompa meccanica per eliminare l'ossigeno, successivamente è stata ripristinata l'atmosfera di argon. Raggiunti i 320°C si è dissolto il CdO rossastro e la soluzione è diventata incolore e limpida. Il pallone è stato lasciato raffreddare fino al raggiungimento di 275°C. In un altro pallone sono stati introdotti tri-ottilfosfina$_{(s)}$ scaldato per renderlo liquido ( TOP, 2.4~mL, 0.831~g/mL, 2.0~g, 370.64~g/mol, 5.4~mmol, Aldrich, 90\% ) e selenio$_{(s)}$ ( Se, 0.0418~g, 0.529~mmol, Aldrich, 99.99\%). Questo è stato scaldato a 45°C ed agitato per mantenere una dispersione. Il contenuto del secondo pallone è stato trasferito in una singola rapida iniezione nel primo pallone. La temperatura è calata fino a 250°C ma è poi stata riportata a 270°C. Il riscaldamento è stato mantenuto per 5 minuti dopo i quali è stato lasciato il pallone con la soluzione rosso cupo al buio per alcune ore a temperatura ambiente. Infine la soluzione è stata conservata a -20°C per il successivo scambio di leganti mentre un'aliquota è stata utilizzata per la caratterizzazione.

Le nanoparticelle in quest'ultima aliquota sono state precipitate dall'ambiente di reazione aggiungendo metanolo disareato (Sigma-Aldrich, 99.8\%). Dopo centrifugazione a 3000rpm per 20 minuti il surnatante è risultato incolore perciò è stato scartato e dell'altro metanolo è stato aggiunto disperdendo il solido. La precipitazione è stata ripetuta 3 volte. Il precipitato è stato seccato con un flusso molto lieve di azoto e quindi ridisciolto in toluene distillato allo scopo di eseguire la caratterizzazione. La soluzione rossa (e verde se osservata sotto illuminazione UV) così ottenuta è stata filtrata con filtro da siringa da 0.2$\mu$m. 

Su questa è stata svolta l'analisi UV-vis, di fotoluminescenza, FT-IR e di \emph{Dynamic Light Scattering}. Dopodiché è stato aggiunto metanolo disareato per riprecipitare le nanoparticelle che sono state centrifugate a 3000~rpm per 20 minuti e seccate con flusso di azoto. È stato successivamente aggiunto CDCl$_3$ (Euriso-Top), un buon solvente per le NPs passivate con TOPO e TDPA, e la soluzione è stata filtrata su filtro da siringa. Questo campione è stato analizzato 
al $^{31}$P $\{^1H\}$ NMR ed al microscopio elettronico a trasmissione. 

La successione di solubilizzazioni in diversi solventi e riprecipitazioni non aveva tanto una funzione preparativa (purificazione) ma piuttosto si è resa necessaria per effettuare le diverse caratterizzazioni su una quantità di prodotto iniziale troppo piccola per poter effettuare prelievi ripetuti.


\section{Sintesi di \emph{quantum dots} di CdSe passivati con piridina}
\label{sec:CdSe-Py}
In un tubo Schlenk con atmosfera inerte di argon sono state poste nanoparticelle di CdSe passivate con TOPO e TDPA (1.348g) e piridina distillata (5~mL). La soluzione è stata agitata per 6 ore a 70°C (con tempi e temperature inferiori non sono stati ottenuti prodotti precipitabili in {\itshape n}-esano (Sigma-Aldrich, puriss.) né in alcuni altri solventi nei quali le nanoparticelle possono perdere la passivazione ed aggregarsi non risultando più solubili in MeOH). Con l'aggiunta di abbondante {\itshape n}-esano le NPs sono state precipitate ed il surnatante è stato scartato dopo centrifugazione a 3000~rpm per 20 minuti. Una parte delle NPs sono state disciolte in \ce{CDCl3} e caratterizzate all' $^1$H-NMR\@. Il solido è stato ripreso in poca piridina, riprecipitato in {\itshape n}-esano e centrifugato a 3000~rpm per 20 minuti; questa operazione è stata ripetuta 5 volte. 

Quindi si è registrato uno spettro $^{31}$P \{$^1$H\} NMR\@. Una ulteriore purificazione è stata eseguita aggiungendo poche gocce di cloroformio e toluene (3~mL) e quindi precipitando con {\itshape n}-esano e centrifugando a 4000rpm per 20 minuti. 
I \emph{quantum dots} di CdSe passivati con piridina e TDPA sono stati seccati con un leggero flusso di gas inerte e quindi sciolti in piridina (0.5~mL) e cloroformio (4.5~mL). Dopo ultrasonicazione per 5 minuti con sonda a ultrasuoni la soluzione è stata centrifugata per 20 minuti a 3000~rpm. Il surnatante è stato filtrato su filtro per siringa da 0.2$\mu$m. 

Questa frazione essiccata è stata ridisciolta in \ce{CDCl3} e caratterizzata all' $^1$H-NMR, al $^{31}$P \{$^1$H\} NMR, all'FT-IR, all'UV-vis ed alla fotoluminescenza. È stata registrata anche una immagine al microscopio elettronico a trasmissione.
\section{Sintesi di poli(3-esiltiofene) regioregolare terminato allile/Br (3)}

In un pallone a tre colli protetto dalla luce è stata ottenuta, e mantenuta per tutta la procedura, l'atmosfera inerte di azoto. Vi è stato versato THF distillato (7~mL) ed il monomero 2,5-dibromo-3-esiltiofene$_{(l)}$ ({\bf{1}}) (1.50~mL, 1.521~g/mL, 2.28~g, 326.09~g/mol, 7.00~mmol) previamente passato all'evaporatore rotante per rimuovere eventuali tracce di solvente. Il pallone è stato posto sotto vuoto da pompa meccanica per eliminare l'eventuale ossigeno disciolto nei liquidi, successivamente è stata ripristinata l'atmosfera di azoto. Utilizzando la siringa ipodermica vi è stato aggiunto il reattivo di Grignard t-butilmagnesio cloruro (7.0~mL, 1.0M in THF, 7.0~mmol, Aldrich). Il pallone è stato lasciato sotto agitazione a 45°C per 4 ore. Una volta raffreddato vi si è aggiunto THF distillato (50~mL) ed il catalizzatore 1,3-bis(difenilfosfino)propano nickel(II) cloruro$_{(s)}$ (0.0630~g, 542.05~g/mol, 0.116~mmol, Aldrich). La soluzione è stata lasciata 20 minuti sotto agitazione al riparo dalla luce. È quindi stato aggiunto allilmagnesio bromuro (1.75~mL, 1.0M in \ce{Et2O}, 
1.75~mmol, Aldrich) ed è agitata la soluzione per 5 minuti. Infine il contenuto del pallone è stato versato goccia a goccia in metanolo (0.75L, Carlo Erba, per analisi) allo scopo di precipitare il polimero formato. La dispersione è stata lasciata sotto agitazione per una notte. 

Il giorno successivo il tutto è stato filtrato attraverso un ditale di cellulosa ed il polimero così raccolto è stato estratto con metanolo e poi con {\itshape n}-esano in Soxhlet fino al termine della colorazione dei solventi uscenti. Il polimero contenuto nel ditale è stato raccolto utilizzando come solvente nell'estrattore Soxhlet del cloroformio distillato. Questa operazione è stata interrotta più volte per sostituire il \ce{CHCl3} contenente il polimero con \ce{CHCl3} distillato. Ciò è stato fatto per minimizzare il danno che può derivare al polimero dal riscaldamento. All'interno del ditale si è osservato il permanere di residuo rosso, probabilmente il catalizzatore. Il polimero ottenuto è stato seccato prima in evaporatore rotante e poi ponendolo sotto vuoto da pompa meccanica. 

Sono stati così ottenuti 0.3897~g di poli(3-esiltiofene) regioregolare \emph{head-tail-head-tail} terminato con un gruppo allile ed un atomo di bromo ({\bf 3}) come solido rosso scuro, con una resa del 33\%.

Una parte del polimero è stata disciolta in \ce{CDCl3} per eseguire la caratterizzazione $^1$H-NMR.



\section{Sintesi del dietil estere dell'acido (4-etinilfenil) fosfonico (5)}
In un tubo Schlenk con atmosfera inerte di azoto è stato versato il catalizzatore \ce{PdCl2(PPh3)2}$_{(s)}$ (0.1329~g, FW=701.90~g/mol, 0.1893~mmol) sciolto in pochi mL di etere dietilico. L'etere dietilico è stato rimosso utilizzando il vuoto da pompa meccanica. È stato aggiunto dietil (4-bromofenil)fosfonato$_{(s)}$ ({\bf 4}) (1.5012~g, 293.09~g/mol, 5.1219~mmol)
, trietilammina (10~mL) e, utilizzando una siringa ipodermica, etiniltrimetil silano$_{(l)}$ (0.83~mL, 0.71~g/mL, 0.59~g, 98.22~g/mol, 6~mmol). 
Infine è stato aggiunto CuI$_{(s)}$ ( 0.0202~g, 190.45~g/mol, 0.1061~mmol) sciolto in trietilammina$_{(l)}$ (5~mL) precedentemente disareato tramite vuoto. È stato collegato il tubo Schlenk alla pompa meccanica da vuoto per eliminare le tracce di ossigeno. Il tutto è stato lasciato sotto agitazione a 90°C per una notte.

La soluzione è stata quindi ridotta di volume utilizzando il vuoto da pompa meccanica. 
È stato aggiunto {\itshape n}-esano e filtrato il surnatante su filtro Gooch di vetro. Il solido rimasto nel tubo Schlenk è stato lavato con una miscela di {\itshape n}-esano  e tetraidrofurano 1:1 fino a decolorazione del solido (fino al persistere della colorazione gialla del catalizzatore). Anche il solvente di lavaggio è stato filtrato su filtro di vetro ed aggiunto alla precedente aliquota filtrata. Da questa soluzione è stato rimosso il solvente all'evaporatore rotante ottenendo un liquido marrone viscoso. Vi è stato aggiunto metanolo anidro (10~mL) e, goccia a goccia, una soluzione acquosa satura di KOH (0.78~g di KOH, 14~mmol, Carlo Erba, per analisi). Il tubo è stato lasciato sotto agitazione protetto dalla luce per 4 ore. La soluzione così ottenuta è stata estratta in imbuto separatore con acqua e \ce{CH2Cl2}. La fase organica è stata anidrificata con solfato di sodio anidro mentre la fase acquosa è stata scartata. Il solvente è stato rimosso utilizzando l'evaporatore rotante. Il legante è stato purificato su colonna impaccata con silice (80~g, Fluka gel di silica 60) utilizzando come eluente una miscela di {\itshape n}-esano  e THF 60:40. Le frazioni raccolte sono state analizzate con TLC\@. Le frazioni in cui si è osservata quasi esclusivamente la macchia del legante fosfonico desiderato sono state unite ed è stato rimosso il solvente con evaporatore rotante. 
 
Sono stati così ottenuti 0.1510~g di legante dietil estere dell'acido (4-etinilfenil) fosfonico ({\bf 5}). Il prodotto è stato conservato in {\itshape n}-esano  a -20°C. 

\section{Sintesi di poli(3-esiltiofene) terminato allile/legante fosfonico protetto (6)}
Il poli(3-esiltiofene) terminato allile/Br \n{3} (0.2g) è stato posto in un tubo Schlenk in cui è stata fatta atmosfera inerte di azoto. È stato aggiunto il legante \n{5} (0.0143~g, 238.22~g/mol, 60.0~$\mu$mol), il catalizzatore \ce{PdCl2(PPh3)2}$_{(s)}$ (0.0040~g, 5.7~$\mu$mol) e tetraidrofurano distillato (9~mL).
La soluzione è stata posta sotto vuoto da pompa meccanica per eliminare gas disciolti.
Infine è stato aggiunto CuI$_{(s)}$ (0.0019~g, 10~$\mu$mol) sciolto in trietilammina$_{(l)}$ (6~mL) precedentemente degassata. Dopo 3 cicli di vuoto azoto il tubo Schlenk è stato lasciato sotto agitazione al riparo dalla luce a 65°C per una notte.

Il polimero è stato quindi precipitato in metanolo e purificato con metanolo in estrattore Soxhlet. Sostituendo il metanolo con \ce{CHCl3} è stato raccolto il polimero {\bf 6}. Nell'eseguire la raccolta in \ce{CHCl3} si è limitato il tempo di permanenza del polimero in cloroformio bollente rimpiazzando più volte il solvente con cloroformio distillato puro. Il prodotto {\bf 6} è stato seccato all'evaporatore rotante e per vuoto da pompa meccanica. Sono stati ottenuti 0.21~g di solido rosso scuro. È stato registrato uno spettro di \n{6} all'FT-IR\@. Una porzione è stata sciolta in \ce{CDCl3} per eseguire la caratterizzazione $^1$H-NMR, $^{13}$C-NMR, $^{31}$P \{$^1$H\} NMR ed HMQC ($^1$H e $^{13}$C NMR bidimensionale). 


\section{Sintesi di poli(3-esiltiofene) terminato allile/legante fosfonico (7)}
Il poli(3-esiltiofene) terminato allile/legante fosfonico protetto \n{6} (0.14g) è stato posto in un tubo Schlenk in atmosfera inerte di azoto e sciolto in \ce{CH2Cl2}$_{(l)}$ (15~mL). Si è aggiunto trietilammina$_{(l)}$ (1.5~mL). Prestando particolare attenzione a minimizzare il contatto con l'aria ed utilizzando una siringa ipodermica è stato versato nel tubo Schlenk trimetilbromosilano$_{(l)}$ (0.5~mL, 1.16~g/mL, 153.09~g/mol, 4~mmol, Aldrich, 97\%). 
Il recipiente è stato lasciato sotto agitazione a temperatura ambiente per una notte. Il giorno successivo è stato scaldato a riflusso per 2 ore ed è stato trasferito in metanolo (0.3L) per la precipitazione. Il precipitato è stato raccolto filtrando su filtro di carta, recuperato lavando il filtro con cloroformio e seccato all'evaporatore rotante e sotto vuoto da pompa meccanica. Sono stati così ottenuti 0.1738~g di solido rosso contenente impurezze bianche (probabilmente prodotti di reazione del trimetilbromosilano con dell'acqua presente nel metanolo utilizzato). Una porzione è stata sciolta in \ce{CDCl3} ed è stata caratterizzata $^1$H-NMR, $^1$H-NMR a 35°C, $^{13}$C-NMR, $^{31}$P \{$^1$H\} NMR a 40°C. È stato registrato uno spettro FT-IR, uno di assorbimento UV-vis ed un'analisi di fotoluminescenza.

\section[Sintesi di \emph{quantum dots} di CdSe passivati con polimero]{Sintesi di \emph{quantum dots} di CdSe passivati con poli(3-esiltiofene) terminato allile/legante fosfonico}
Al polimero {\bf 7} (0.21g, dei quali una piccola parte ottenuti da una seconda preparazione identica a quella qui riportata) disciolto in cloroformio (0.5~mL) sono state aggiunte le nanoparticelle (0.183g) ottenute nella Sezione~\ref{sec:CdSe-Py} seccate e ridisciolte in cloroformio (1~mL). La soluzione così ottenuta è stata lasciata a temperatura ambiente per 2 giorni. Questo prodotto è stato caratterizzato con FT-IR, UV-vis, fotoluminescenza, TEM e $^{31}$P \{$^1$H\} NMR.
